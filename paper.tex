% Use line with ``preprint'' to get double-spaced
% preprint version. Use line without preprint to get
% version to submit to PRE.

\documentstyle[aps,twocolumn,prl,epsf]{revtex}
%\documentstyle[aps,preprint,prl,pifont,citesort,epsf]{revtex}

\begin{document}

\draft

\title{Sub-10 Hz dark resonance at the cesium clock frequency induced by an optical frequency comb laser}

\author{
Tsung-Han Wu\footnote[*]{Arisona State}, Yan-Long Peng and Wang-Yau Cheng\cite{wangyau-email}
}

\address{
Institute of Atomic and Molecular Sciences, Academia Sinica, Taiwan, R.O.C.
}

\date{\today}

\maketitle

% Max of 600 characters for PRL, Rapid Comm abstracts.

\begin{abstract}
By precisely controlling the comb laser repetition rate, we
resolved a dark state of exceptionally narrow linewidth (5.6~Hz)
in cesium gas buffered by neon atoms. Moreover, the serious
pressure shift of the cesium hyperfine splitting (clock frequency)
expected from the literature was not observed. We theoretically
interpreted our experimental data and point out that this narrow
resonance can potentially be used not only to enhance the frequency 
discrimination of a microwave clock to the sub-10~Hz level, but also to
provide a simple optical clockwork linking the microwave time standard
to the optical frequency.
\end{abstract}

\pacs{
%39.30.+w,  %% Does not exists! What was it?
32.70.Jz,  % Line shapes, widths, and shifts 
32.50.+d   % Fluorescence, phosphorescence (including quenching)
% 32.80.Qk 	Coherent control of atomic interactions with photons  
}


\narrowtext
Ultimate atomic wavefunction control needs resolution at hyperfine level and needs highly coherent electromagnetic waves to build up stable quantum interference. Quantum interference established by an optical frequency comb laser was recently demonstrated to be a novel approach for the ultimate wavefunction control~\cite{Stowe2008}, although precise applications were not discussed. In this work, we observed a comb-laser-induced mixed quantum state, which is often called a -Y´dark state¡ generated by coherent population trapping (CPT) [ii]. We also point out a potential clockwork application via direct transfer of the microwave frequency standard to an optical frequency. Comb-laser-based Cs CPT clock, once developed with a high Q-value (), would be generalizable to extremely wide-band wavelength regions [iii,iv] due to the high peak power, and should therefore have much wider application than a CW-based CPT clock.
    
 Probing a dark resonance by a mode-locked pulse train was first demonstrated by Arissian and Diels [v], who resolved a dark state of 627 kHz linewidth with hot Rb atoms. They concluded from a theoretical analysis that stabilizing the laser frequency was not necessary to prepare their dark state. We confirm their observations here using our unique comb laser system [vi], which allows for directly monitoring and controlling the mode frequency (fn) [6,vii] without interrupting the repetition rate locking. However, we observed a 105 order of magnitude narrower CPT linewidth in 8.7 kPa Ne buffer gas and we found, somewhat surprisingly, that our narrow dark-state position was not sensitive to buffer-gas pressure in an optical-thick medium. We were also aware that their modeling was only applicable in some situations, e.g., when the spectral linewidth of the intermediate states is much larger than the repetition rate or the hyperfine splittings that three-level model could fairly cope with. 

     To resolve the ultra-narrow dark resonance and to verify the role of mode frequency, we constructed a comb laser system [8] with the block diagram shown at the top of Fig. 1. The mode frequency was monitored by a reference laser whose frequency instability (f) was 160 Hz @ 60 second [9]. Mode-frequency (fn) locking was realized by horizontally shifting the pump beam with a piezoelectric transducer (PZT) [8]. The repetition rate (frep) was phase locked against an synthesizer [viii] whose time base was referred to a satellite-based 5-MHz frequency standard via Loran-C. An additional [10] function generator was employed to improve the measurement resolution to 10 Hz. To ensure ´orthogonal control¡ between the mode-frequency (fn) and the repetition rate (frep), we dithered fn by the PZT indicated in Fig. 1, with the dither width ranging from 1 to 10 MHz, while simultaneously recording the maximum instability of the repetition rate locking. The result of this orthogonality inspection shows (inset of Fig. 1) that there is no correlation between frep and fn-dither-width at our 4-mHz measurement uncertainty. Here, as usual,, fn and frep are related by fnfrepnwhere n denotes the mode number; frep is the repetition rate; and  is the overall phase difference between successive pulses. The lower part of Fig. 1 shows the block diagram for retrieving the CPT signal. The 40-fs comb pulse, which was sent into the spatial light modulator (SLM) system, had 200-mW average power with a repetition rate of 91.92631770 MHz = (clock frequency)/100. The spatial light modulator extracted the 852 nm component from all other comb modes; so the scattered background light from other wavelengths could be minimized. Consequently, 0.2-nm bandwidth comb modes with 140W/cm2 average power were sent into a Cs cell, with a wrapping of three layers of -metal and a solenoid coil on the inner shell of the -metal. To reduce the spin-spin-exchange collisions as well as transit-time broadening [ix], we tried different Cs vapor pressures and different buffer gases. A photomultiplier tube (PMT) detected fluorescence from Cs 6P3/2 6S1/2 transition. The length of the light path from the input surface of the cell to the location of the PMT was around 3.8 cm. We chopped the laser beam with a 500 Hz chopping frequency and measured the small reduction of fluorescence with a lock-in amplifier. An acoustic-optical modulator (AOM) was used to stabilize the laser power to the noise level of 50-nW. Power fluctuations at W level will obscure the CPT signal.

Fig. 2 shows experimental results for a mixed buffer gas of 1.5 kPa N2 and 1.5 kPa He. Fig. 3 shows results for 8.7 kPa Ne buffer gas. In Figs. 2 and 3, the estimated spectral linewidth of a one-photon resonance () is 260 MHz and 750 MHz (D2 line), respectively, i.e., larger than the comb laser repetition rate (92 MHz). A CPT linewidth of 706 Hz (Gaussian fitting) with 4% contrast was found at the laser repetition rate tuning over 1/100th of the clock frequency (Fig. 2). Sampling time for each data point was 30 seconds. Note that our Cs cell was heated to 1000.001 0C to increase the optical density and was, therefore, no longer optically thin. We attribute the exceptionally narrow linewidth to the line-narrowing effect that occurs in an optically thick medium [x,xi] and to the in-phase comb modes-atom interactions that leads to the reduction of FM-to-AM noise [xii]. 

The narrowest CPT linewidth we measured was 5.20.6 Hz (Fig. 3), which is 20-fold narrower than what has been achieved with a CW laser [xiii]. In Fig. 3, the scatter in the data points around the transparent dip is mainly due to the instrumental limitation [10] on the accuracy of repetition rate. Denser data points were obtained when a high resolution function generator [10] was employed, as shown in the inset of Fig. 3. By simultaneously monitoring the CPT-signal and the mode-frequency, we found that the CPT signal was not sensitive to mode frequency, as the conclusion of Ref. 7. However, when the fn was dithered with a frequency faster than 10 kHz and a width larger than 1 MHz, the CPT signal could become indistinct, since severe FM-to-AM noise could be larger than the signalError: Reference source not found [14], though this situation did not in general occur in our well-controlled laser system. The interesting feature that a comb-laser based CPT signal depends on only one parameter, i.e., only on the repetition-rate, is encouraging news for simplifying any future repetition-rate based CPT standard. 

One puzzling phenomenon remains in this work, namely, no pressure shift was observed in our experiments, even though a pressure shift of 43-kHz was predicted under the conditions of 8.7 kPa Ne buffer gas at room temperature [Error: Reference source not found15]. This lack of obvious pressure shift is more encouraging news for future CPT-clock devices.

     Since no theory has been reported on the subject of comb-laser-based CPT, we interpret our experimental results by employing a set of Bloch equations with density matrix elements similar to Ref. 2, except that the Rabi frequency changes rapidly with time, the dimension of the density matrix is extended to 44 for our four-level system, and Adomian’s decomposition method [xiv] was adopted for solving the coupled partial differential equations in the time domain. To simplify the time-domain calculation, we let the pulse carrier frequency (fc) coincide with one comb mode (fn), and this simplification was examined and shown to have no influence on the conclusions of this letter. The corresponding level diagram is depicted on the right of Fig. 4 (a), where ij denotes the matrix elements of the four-by-four density matrix; ij is the level decay rate of excited states; 1, 2 are the decoherent rate of the diagonal and off-diagonal terms between level 3 and 4, respectively; and ij is the i j level transition frequency (i,j=1~4). Note that, 11+22, the total upper-level population, can be taken as the signature of ground state coherence when 34 is built up. The level diagram in Fig. 4 (a) can be applied to both the D1 and D2 lines of the Cs atom. The other hyperfine levels of the D2 line, namely, F’=2 and F’=5, are not relevant in building up the dark state, and are considered as part of the ground-state decoherence rate, i.e., as levels contributing to 1 and 2. All the data used in the simulation of this letter were taken from reference [xv]. Fig. 4 shows that the CPT signal exhibits a carrier-frequency dependence, which is different from the conclusion in Ref. 7. The main cause of this difference is that the corresponding spectral width () of the one-photon transitions is assumed as much smaller than the repetition rate, , as in the case of a Cs MOT or atomic beam (5.2 MHz). In contrast, when the simulation is based on our experimental conditions in Fig. 3; that is,  =750 MHz; laser average power = 140W/cm2 and pulse width = 1 picosecond after the spatial light modulator, the CPT lineshape becomes insensitive to the carrier frequency of comb pulse, which is consistent with the conclusion in Ref. 5. Fig. 5 illustrates the concept of -Y΄carrier-frequency-dependent quantum interference‘ as the frep in Fig. 4 is kept at 1/100th clock frequency. In other words, the ΄bright bump‘ indicated by the black curve of Fig. 5 shows an increase in (11+22). This interesting phenomenon cannot occur in a three-level system. Our interpretation is illustrated in the inset of Fig. 5: the two CPT channels, namely, F=3F’=3F=4 (blue path at the lower right) and F=3F’=4F=4 (red path at the lower right), can interfere with each other, which leads to a small -Y΄bright bump‘ in the dark state. The highest bump (fc=8.7 MHz) happens when the comb laser carrier frequency is tuned such that the corresponding Rabi frequencies associated with the F=4F’=4 and F=4F’=3 transitions are the same but with opposite phase.  A similar situation happens for the other bump (f=-37.8 MHz). This fc-dependent -Y΄bright bump‘ offers a chance to further stabilize the carrier frequency by locking.the repetition rate to the bump center, and thus the optical frequency is connected to microwave standard simply using a Cs cell.    

The authors are deeply indebted to Dr. Jyhpyng Wang for encouraging and kindly amending this article.  The author also thanks Dr. Ming-Sheng Chang, Dr. Ying-Chen Cheng, and Professor Jow-Tsong Shy for valuable comments, and Dr. Jon Hougen for the help with English corrections. We are grateful for the funding support of National Science Council in Taiwan with the project of NSC 94-2112-M-001-022-MY3.



\begin{references}

\bibitem[\dagger]{wangyau-email} E-mail: wycheng@gate.sinica.edu.tw

\bibitem{Stowe2008} M.~C. Stowe {\it et~al.}, Phys. Rev. Lett. {\bf 100}, 203001 (2008).

\bibitem{Vanier2005} J. Vanier, Appl. Phys. {\bf B 81}, 421 (2005).

\bibitem{Gohle2005} C. Gohle {\it et~al.}, Nature {\bf 436}, 234 (2005).

\bibitem{Dudley} J.~M. Dudley {\it et~al.}, Rev. Mod. Phys. {\bf 78}, 1135 (2006).

\bibitem{Arissian2006} L. Arissian and J.–C. Diels, Opt. Comm. {\bf 264}, 169 (2006).

\bibitem{Cheng2008} W.-Y. Cheng {\it et~al.}, Appl. Phys. B {\bf 92}, 13 (2008).

\bibitem{Cheng2007} C.-Y. Cheng {\it et~al.}, Opt. Lett. {\bf 32}, 536 (2007).

\bibitem{Synth} Agilent 8644A synthesizer and Agilent 81150 function generator, respectively.

\bibitem{Camparo2007} J. Camparo, Phys. Today {\bf 60}, 33 (2007).

\bibitem{Lukin1997} M.~D. Lukin {\it et~al.}, Phys. Rev. Lett. {\bf 79}, 2959 (1997).

\bibitem{Hockel2009} D. Hockel {\it et~al.}, Appl. Phys. B {\bf 94}, 429 (2009).

\bibitem{Camparo} J.~C. Camparo and J.~G. Coffer, Phys. Rev. A {\bf 59}, 728 (1999).

\bibitem{Knappe2001} S. Knappe {\it et~al}, JOSA B {\bf 18}, 1545 (2001).

\bibitem{Kao2005} Y.~M. Kao {\it et~al}, Phys. Rev. E {\bf 72}, 066703 (2005).

\bibitem{Steck2009} Daniel A. Steck, ΄Cesium D line data‘, available online at http://steck.us/alkalidata (revision 2.1.2, 12 August 2009).




% \bibitem{spiral-refs} A.~T.\ Winfree, Chaos {\bf 1}, 303
% (1991); R.~A.\ Gray and J.~Jalife, Int. J. Bifur. Chaos {\bf
% 6}, 415 (1996).

% \bibitem{heart-refs} 
% J.~J. Lee {\it et~al.}, Circulation Research {\bf 78}, 660
% (1995); A. Garfinkel {\it et~al.}, Journal of Clinical
% Investigation {\bf 99}, 305 (1997).

% \bibitem{baer-refs}
% M. B\"ar and M. Eiswirth, Phys.\ Rev.~E {\bf 48}, R1635
% (1993); M. B\"ar {\it et~al.}, Chaos {\bf 4}, 499 (1994).

% \bibitem{Hildebrand95} M.~Hildebrand, M. B\"ar, and
% M.~Eiswirth, Phys. Rev. Lett. {\bf 75}, 1503 (1995).

% \bibitem{Bayly93}
% P.~V. Bayly {\it et~al.}, Journal of
% Cardiovascular Electrophysiology {\bf 4}, 533 (1993).

% \bibitem{Karma}
% A. Karma, Chaos {\bf 4}, 461 (1993).

% \bibitem{Panfilov95}
% A.~V. Panfilov, Science {\bf 270},  1224  (1995).

% \bibitem{control-refs}
% %\bibitem{Ott90,Garfinkel92,Hu95}
% E.~Ott, C.~Grebogi, and J.~A. Yorke, Phys.\ Rev.\ Lett.~{\bf
% 64}, 1196 (1990); A. Garfinkel, M.~L. Spano, W.~L. Ditto,
% and J.~N. Weiss, Science {\bf 257}, 1230 (1992); G.~Hu,
% Z.~Qu, and K.~He, Int.~J.\ Bifurcations and Chaos {\bf 5},
% 901 (1995).

% \bibitem{Winfree91}
% M. Courtemanche and A. Winfree, Int. J. Bifurc. Chaos, {\bf 1}, 431 (1991).

% \bibitem{Barkley91}
% D. Barkley, Physica D {\bf 49},  61 (1991).

% \bibitem{Parker89} T.~S. Parker and L.~O. Chua, {\em
% Practical Numerical Algorithms for Chaotic Systems}
% (Springer-Verlag, New York, 1989).

% \bibitem{strain-ms-thesis97} M.~Strain, Master's thesis,
% Duke University (December 1997), ``Spatiotemporal dynamics
% of excitable and intermittent systems''. 

% \bibitem{transient-refs}
% J.~Crutchfield and K.~Kaneko. Phys. Rev. Lett. {\bf 60}, 2715 (1988).
% B.~I. Shraiman, Phys. Rev. Lett. {\bf 57},  325  (1986);
% A. Wacker, S. Bose, and E. Sch\"oll, 
% Europhysics Letters {\bf 31},  257 (1995).

% \bibitem{Winfree95}
% A.~T. Winfree, Science {\bf 270},  1223  (1995).

% \bibitem{Cross93}
% M.~C. Cross and P.~C. Hohenberg, 
% Rev. Mod. Phys. {\bf 65},  851  (1993).

% \bibitem{OHern96} C. O'Hern, D. Egolf, and H. Greenside,
% Phys. Rev.~E {\bf 53}, 3374 (1996).

% \bibitem{ZhangHolden95} H. Zhang and A.~V. Holden, Chaos,
% Solitons and Fractals {\bf 5}, 661 (1995).

% \bibitem{Garfinkel97} A. Garfinkel {\it et~al.}, Journal of
% Clinical Investigation {\bf 99}, 305 (1997).

% \bibitem{Egolf97} D.~A. Egolf,
% Available on the LANL pre-print server (1997).

% \bibitem{embedding-refs} F. Ravelli and R. Antolini, in {\em
% Nonlinear wave processes in excitable media}, edited by
% A.~V. Holden (Plenum Press, New York, 1991), pp.\ 335--341;
% A.~L. Goldberger, V. Bhargava, B.~J. West, and
% A.~J. Mandell, Physica D {\bf 19}, 282 (1986); D. Kaplan and
% R. Cohen, Circulation Research {\bf 67}, 886 (1990);
% F.~X. Witkowski {\it et~al.}, Phys. Rev. Lett. {\bf 75},
% 1230 (1995).

\end{references}

\newpage

% \begin{figure}   % fig 1
% \caption{Density plot at time $t=500$ of the slow field
% $v(t,x,y)$ for a spatiotemporal chaotic state with
% $31$~spiral defects present. Dark and light regions
% correspond to values less or greater than the
% value~$v^*=0.484$ corresponding to the unstable fixed point;
% the field values span the range $v \in [0,a-b]$.  Parameter
% values were $\epsilon = 0.074$, $a = 0.84$, $b = 0.07$, $L =
% 50$, $\Delta{x} =0.5$ and~$\Delta t = 0.0037$. }
% \label{fig:pattern}
% \end{figure}

% \centerline{\epsfysize=3in \epsfbox{Pattern.eps}}


\end{document}
